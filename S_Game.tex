
\documentclass[journal]{IEEEtran}


%%------------------------------------------------------
%% Packages
%%------------------------------------------------------
\usepackage[T1]{fontenc}           %% Codificação de caracteres
\usepackage[utf8]{inputenc}       
\usepackage[dvips]{graphicx}       %% para a macro includegraphics 
\usepackage[english,brazil]{babel} %% PT_BR e EN (o último define a prioridade no arquivo)
\usepackage{pgf}                   %% macro para criar gráficos
\usepackage{epsfig}                %% or use the epsfig package if you prefer to use the old commands
\usepackage{graphics}              %% use the graphics package for simple commands
\usepackage{graphicx}              %% or use the graphicx package for more complicated commands
\usepackage{epstopdf}              
\usepackage{float}                 
\usepackage{eqparbox}              
\usepackage{hyphenat}              
\usepackage{hyperref}              

% \usepackage{showframe} %% just for the example

% \usepackage[sort,compress]{cite} %% disable if natbib package is activated
\usepackage[numbers,sort&compress,square]{natbib} %% e.g., [2-5]



%%------------------------------------------------------
%% Definitions
%%------------------------------------------------------

\hyphenation{op-tical net-works semi-conduc-tor}

%% Can use something like this to put references on a page
%% by themselves when using endfloat and the captionsoff option.
\ifCLASSOPTIONcaptionsoff
  \newpage
\fi


%%----------------- Definindo as variáveis com números
\makeatletter
%
\newcommand{\prenome}{\afterassignment\prenome@aux\count0=}
\newcommand{\prenome@aux}{\csname prenome\the\count0\endcsname}
%
\newcommand{\nomedomeio}{\afterassignment\nomedomeio@aux\count0=}
\newcommand{\nomedomeio@aux}{\csname nomedomeio\the\count0\endcsname}
%
\newcommand{\sobrenome}{\afterassignment\sobrenome@aux\count0=}
\newcommand{\sobrenome@aux}{\csname sobrenome\the\count0\endcsname}
\makeatother
%%%%%

%%----------------- Configurações de hyperlinks
%% Não decorado, sem destaque
\hypersetup{
  colorlinks=false,
  pdfborder={0 0 0},
}


%%----------------- Título
\title                                                {Space Game}

\newcommand{\emailautor}                              {yasmin25kioko@gmail.com}

\newcommand{\siglaRevista}                            {UFSC}

\newcommand{\Revista}                                 {Universidade Federal de Santa Catarina (UFSC)}



%%----------------- Autor Principal começo
\newcommand{\prenomePrincipal}                        {Yasmin}
\newcommand{\nomedomeioPrincipal}                     {Kioko Shimabuku}
\newcommand{\sobrenomePrincipal}                      {da Silva}


%%----------------- Apenas um autor no final
\author{\IEEEauthorblockN{\prenomePrincipal~\nomedomeioPrincipal~\sobrenomePrincipal\IEEEauthorrefmark{1}}

\IEEEauthorblockA{\IEEEauthorrefmark{1}Universidade Federal de Santa Catarina (UFSC)}% <-this % stops an unwanted space
%%
\thanks{\Revista. Correspond\^encia ao autor: \prenomePrincipal~\nomedomeioPrincipal~\sobrenomePrincipal~(email: \emailautor).}}


%%------------------------------------------------------
%% Cabeçalho
\markboth{\MakeUppercase{\Revista}}%% acentuação indireta
%% Apenas um autor:
{\sobrenomePrincipal: \MakeUppercase{\Revista}}
%% Mais de um autor:
% {\sobrenomePrincipal: \MakeLowercase{\textit{et al.}}: \Revista}


%%------------------------------------------------------
%% Abstract
\IEEEtitleabstractindextext{

  {\selectlanguage{brazil}
    \begin{abstract}
     O jogo Space Game foi elaborado pensando na missão Apollo 11, realizada em 1969, na qual os astronautas Neil Armstrong, Michel Collins e Edwin Aldrin realizaram o primeiro pouso em solo lunar.
    \end{abstract}
    
    
    %%----------------- Keywords
    \renewcommand\IEEEkeywordsname{Palavras-chave} %% Palavras-chave ao invés de 'Index Terms'
    \begin{IEEEkeywords}
    Jogo, Space Game, Ncurses.
    \end{IEEEkeywords}
  }
}


\begin{document}


%%------------------------------------------------------
%% Inserção de informações
\maketitle
\IEEEdisplaynontitleabstractindextext
\IEEEpeerreviewmaketitle


%%------------------------------------------------------
%% Section
\section{Introdução}

\IEEEPARstart{O} jogo Space Game foi criado na intenção de fazer um indivíduo conseguir coletar um número máximo de estrelas, a fim de desbloquear o número de fases e velocidades disponíveis. Contudo, conforme o número da pontuação aproxima-se de uma fase, a velocidade do jogo também aumenta, dificultando a vitória. 


%%------------------------------------------------------
%% Section
\section{Metodologia}

Foram utilizadas as bibliotecas – stdio, ncurses, time, stdlib, unistd – a fim de facilitar a programação do jogo. A principal biblioteca utilizada foi a Ncurses, pois permite a manipulação de texto em coordenadas. 

De início, foram declaradas variáveis inteiras e do tipo char para fazer com que o código reconheça os comandos ou os loops.  Em seguinda, funções da biblioteca Ncurses foram iniciadas, para fazer com que as teclas W,A,S,D e ESC sejam reconhecidas, por exemplo. 

Para a formação da interface, foram utilizados mvprintw e move vline, que permitem o desenho de barras limitantes em linhas e colunas específicas. Além disso, para que ocorra a mudança de janela, foi utilizada a função getch, que espera o usuário inserir uma tecla para continuar a leitura do código. 

A fim de tornar o jogo mais interativo, é pedido a escolha da velocidade inicial e a letra inicial do nome do usuário, e ambas as informações são armazenadas em variáveis, int e char respectivamente, para serem utilizadas posteriormente. Após a escolha de um número, ele é utilizado na função velocidade – quanto maior o número, maior a velocidade do jogo. Durante a inclusão do personagem, foi chamada a variável da letra que o usuário digitou e ela é responsável por mover-se na tela. Para movimentação do usuário, foram usadas as teclas W, A, S, D associadas a tabela ASCII, a fim de facilitar a movimentação em diversas formas de teclado. 

Para que o código mantenha-se funcionando foi utilizada uma condição, até que o valor da variável fosse diferente. Dentro do loop, temos variáveis que armazenam valores das posições das coordenadas x e y, da inicial do usuário e da estrela. Conferindo  se a coordenada do usuário é igual a coordenada da estrela, e caso isso aconteça é marcado um ponto e emite um som. Com a finalidade de tornar as estrelas aleatórias, foi utilizada a função rand e com isso, a estreja sempre é gerada dentro do limite das linhas de borda. Para que ocorra a perda do jogo, é necessário que o usuário esteja na mesma posição que as linhas de borda, sendo preciso recomeçar.

Todas as vezes que o usuário consegue um valor de cinco pontos uma imagem é desbloqueada, indicando a passagem de fase. Quando o total de pontos é igual o número programado, finaliza-se o jogo com a vitória.

%%------------------------------------------------------
%% Section
\section{Conclusão}

O jogo Space Game funcionou como esperado, utilizando os conhecimentos das bibliotecas e funções descritas. 


\end{document}
